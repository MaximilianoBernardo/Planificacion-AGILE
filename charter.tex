\documentclass[
11pt, % The default document font size, options: 10pt, 11pt, 12pt
%codirector, % Uncomment to add a codirector to the title page
]{charter} 


% El títulos de la memoria, se usa en la carátula y se puede usar el cualquier lugar del documento con el comando \ttitle
\titulo{Analisis de asistencia a consultas odontológicas y predicción de ausencia de pacientes a los turnos} 

% Nombre del posgrado, se usa en la carátula y se puede usar el cualquier lugar del documento con el comando \degreename
%\posgrado{Carrera de Especialización en Sistemas Embebidos} 
%\posgrado{Carrera de Especialización en Internet de las Cosas} 
\posgrado{Carrera de Especialización en Inteligencia Artificial}
%\posgrado{Maestría en Sistemas Embebidos} 
%\posgrado{Maestría en Internet de las cosas}

% Tu nombre, se puede usar el cualquier lugar del documento con el comando \authorname
% IMPORTANTE: no omitir titulaciones ni tildación en los nombres, también se recomienda escribir los nombres completos (tal cual los tienen en su documento)
\autor{Ing. Maximiliano Jose Bernardo}

% El nombre del director y co-director, se puede usar el cualquier lugar del documento con el comando \supname y \cosupname y \pertesupname y \pertecosupname
\director{Me comunique con un profesor del postgrado, estoy esprando su respuesta}
\pertenenciaDirector{pertenencia} 
\codirector{} % para que aparezca en la portada se debe descomentar la opción codirector en los parámetros de documentclass
\pertenenciaCoDirector{FIUBA}

% Nombre del cliente, quien va a aprobar los resultados del proyecto, se puede usar con el comando \clientename y \empclientename
\cliente{Micaela Flores}
\empresaCliente{Mundo Dental}
 
\fechaINICIO{21 de octubre de 2025}		%Fecha de inicio de la cursada de GdP \fechaInicioName
\fechaFINALPlan{9 de diciembre de 2025} 	%Fecha de final de cursada de GdP
\fechaFINALTrabajo{9 de diciembre de 2025}	%Fecha de defensa pública del trabajo final


\begin{document}

\maketitle
\thispagestyle{empty}
\pagebreak


\thispagestyle{empty}
{\setlength{\parskip}{0pt}
\tableofcontents{}
}
\pagebreak


\section*{Registros de cambios}
\label{sec:registro}


\begin{table}[ht]
\label{tab:registro}
\centering
\begin{tabularx}{\linewidth}{@{}|c|X|c|@{}}
\hline
\rowcolor[HTML]{C0C0C0} 
Revisión & \multicolumn{1}{c|}{\cellcolor[HTML]{C0C0C0}Detalles de los cambios realizados} & Fecha      \\ \hline
0      & Creación del documento                                 &\fechaInicioName \\ \hline
%1      & Se completa hasta el punto 5 inclusive                & {día} de {mes} de 202X \\ \hline
%2      & Se completa hasta el punto 9 inclusive
%		  Se puede agregar algo más \newline
%		  En distintas líneas \newline
%		  Así                                                    & {día} de {mes} de 202X \\ \hline
%3      & Se completa hasta el punto 12 inclusive                & {día} de {mes} de 202X \\ \hline
%4      & Se completa el plan	                                 & {día} de {mes} de 202X \\ \hline

% Si hay más correcciones pasada la versión 4 también se deben especificar acá

\end{tabularx}
\end{table}

\pagebreak



\section*{Acta de constitución del proyecto}
\label{sec:acta}

\begin{flushright}
Buenos Aires, \fechaInicioName
\end{flushright}

\vspace{2cm}

Por medio de la presente se acuerda con el \authorname\hspace{1px} que su Trabajo Final de la \degreename\hspace{1px} se titulará ``\ttitle'' y consistirá en la implementación de un sistema de predicción de asistencia de pacientes. El trabajo tendrá un presupuesto preliminar estimado de 600 horas y un costo estimado de \$ 20000000, con fecha de inicio el \fechaInicioName\hspace{1px} y fecha de presentación pública el \fechaFinalName.

Se adjunta a esta acta la planificación inicial.

\vfill

% Esta parte se construye sola con la información que hayan cargado en el preámbulo del documento y no debe modificarla
\begin{table}[ht]
\centering
\begin{tabular}{ccc}
\begin{tabular}[c]{@{}c@{}}Dr. Ing. Ariel Lutenberg \\ Director posgrado FIUBA\end{tabular} & \hspace{2cm} & \begin{tabular}[c]{@{}c@{}}\clientename \\ \empclientename \end{tabular} \vspace{2.5cm} \\ 
\multicolumn{3}{c}{\begin{tabular}[c]{@{}c@{}} \supname \\ Director del Trabajo Final\end{tabular}} \vspace{2.5cm} \\
\end{tabular}
\end{table}




\section{1. Descripción técnica-conceptual del proyecto a realizar}
\label{sec:descripcion}

Este proyecto es un emprendimiento personal que tiene como cliente a la empresa Mundo Dental, un consultorio odontológico ubicado en San Martín. El objetivo surgió tras dialogar con la jefa del consultorio, Micaela Flores. Durante estas conversaciones, identificamos el problema de las ausencias de los pacientes a los turnos odontológicos. Estas ausencias generan dificultades en la gestión del tiempo y en la facturación del consultorio, ya que se han registrado casos en los que faltan varios o incluso la mayoría de los pacientes en un determinado día, lo que provoca que una jornada laboral resulte improductiva. A partir de estas observaciones, planteamos la hipótesis inicial de que la cantidad de pacientes que se ausentan a los turnos es lo suficientemente significativa como para que, a largo plazo, el consultorio pueda enfrentar serias dificultades económicas.

En función del problema presentado por el cliente, procederemos a recopilar información de los turnos registrados durante el último año. Con estos datos realizaremos un análisis de la asistencia y de la ausencia de los pacientes, con el fin de determinar el volumen de ausencias presente en el conjunto de datos recopilado. Este análisis inicial nos permitirá evaluar la hipótesis planteada al comienzo. Una vez resuelta la incertidumbre respecto al volumen de ausencias, utilizaremos la información obtenida para desarrollar un modelo de machine learning tradicional, cuyo objetivo será predecir la probabilidad de ausencia de un paciente. Además, implementaremos un chatbot de WhatsApp que gestionará y asignará los turnos, y que realizará su seguimiento de forma automática. El propósito de esta herramienta será optimizar el uso de los turnos disponibles por día.

Este proyecto es valorado por el cliente, ya que le permitirá tomar decisiones informadas sobre el futuro de la empresa. Según los resultados obtenidos a partir de la hipótesis inicial, definiremos qué acciones se deberán tomar, determinando si el negocio es viable en su estado actual, si requiere ajustes en su gestión o si resulta inviable con el grupo de clientes actual de la zona. Por otro lado, el chatbot permitirá al cliente administrar de manera más simple los turnos asignados, lo que le brindará más tiempo para realizar otras tareas administrativas. Finalmente, el modelo de predicción de ausencias posibilitará al sector administrativo analizar la factibilidad de sobreasignar turnos en días o momentos específicos.

En la figura \ref{fig:diagrama} se presenta el diagrama en bloques del sistema

\begin{figure}[htpb]
\centering 
\includegraphics[width=.65\textwidth]{./Figuras/diagrama.png}
\caption{Diagrama en bloques del sistema.}
\label{fig:diagrama}
\end{figure}


\section{2. Identificación y análisis de los interesados}
\label{sec:interesados}

\begin{table}[ht]
%\caption{Identificación de los interesados}
%\label{tab:interesados}
\begin{tabularx}{\linewidth}{@{}|l|X|X|l|@{}}
\hline
\rowcolor[HTML]{C0C0C0} 
Rol           & Nombre y Apellido  & Organización 	& Puesto 	\\ \hline
Auspiciante   &        -           &         -     	&     -   	\\ \hline
Cliente       & \clientename       &\empclientename	&     -  	\\ \hline
Impulsor      &        -           &        -      	&     -   	\\ \hline
Responsable   & \authorname        & FIUBA        	& Alumno 	\\ \hline
Colaboradores &         -          &        -      	&     -   	\\ \hline
Orientador    & \supname	       & \pertesupname 	& Director del Trabajo Final \\ \hline
Equipo        &          -         &        -      	&      -  	\\ \hline
Opositores    &          -         &       -      	&      -  	\\ \hline
Usuario final &         -         &       -     	&      -  	\\ \hline
\end{tabularx}
\end{table}


\begin{itemize}
    \item Cliente: {\clientename} es la dueña del consultorio y será la clienta final que use el sistema desarrollado.
    \item Responsable: {\authorname}   es el encargado de llevar adelante el proyecto y velar por los objetivos del mismo.
    \item Orientador: {\supname} es el referente de la arquitectura que se implementará para el proyecto.
\end{itemize}




\section{3. Propósito del proyecto}
\label{sec:proposito}

El proyecto tiene como objetivo desarrollar un modelo predictivo capaz de estimar la probabilidad de asistencia de los pacientes a sus turnos odontológicos, complementado con un agente de inteligencia artificial que funcione como chatbot para la gestión automática de los turnos. Este asistente permitirá interactuar con los pacientes mediante recordatorios y consultas sencillas sobre su asistencia, integrando sus respuestas en el modelo para mejorar la precisión de las predicciones. Con ello se busca optimizar la planificación de la agenda odontológica, reducir el impacto de las inasistencias y proporcionar a los usuarios finales una herramienta práctica, accesible y confiable para la gestión de turnos.


\section{4. Alcance del proyecto}
\label{sec:alcance}

El alcance de este proyecto comprende el desarrollo de un sistema capaz de predecir la asistencia de los pacientes a sus turnos odontológicos y gestionar la interacción con ellos mediante un asistente virtual o \textit{chatbot}. En particular, se incluyen los siguientes elementos:

\begin{enumerate}
    \item Recopilación y unificación de los registros históricos de turnos odontológicos de los pacientes, idealmente cubriendo al menos el último año de actividad.
    \item Limpieza y estructuración de los datos para asegurar consistencia y confiabilidad en el análisis posterior.
    \item Análisis exploratorio de los datos, incluyendo patrones de asistencia y ausencia por día de la semana, distribución mensual y detección de tendencias relevantes.
    \item Desarrollo de un modelo predictivo que estime la probabilidad de asistencia de cada paciente, utilizando algoritmos como \textit{Random Forest}, \textit{Regresión Logística} o \textit{Support Vector Machines} (\textit{SVM}).
    \item Integración de un agente de inteligencia artificial (\textit{IA}) que permita a los pacientes reservar turnos, recibir recordatorios y responder consultas sobre su asistencia mediante plataformas de mensajería (por ejemplo, \textit{WhatsApp} o \textit{Telegram}).
    \item Integración de la información recopilada por el \textit{chatbot} en el modelo predictivo, para mejorar la precisión de las predicciones en tiempo real.
    \item Implementación de una interfaz para el usuario final, que permita ingresar los datos de un turno y recibir la predicción correspondiente.
\end{enumerate}

El presente proyecto \textbf{no incluye}:

\begin{itemize}
    \item La digitalización de registros previos que no se encuentren disponibles en archivos electrónicos legibles.
    \item La gestión administrativa completa del consultorio más allá de la interacción automatizada con los pacientes y la predicción de asistencia.
    \item La integración con sistemas externos distintos a los servicios de mensajería especificados (\textit{WhatsApp}, \textit{Telegram}) o la conexión con otras plataformas de gestión.
    \item Mantenimiento, soporte o actualización continua del sistema luego de su entrega inicial.
\end{itemize}


\section{5. Supuestos del proyecto}
\label{sec:supuestos}

Para el desarrollo del presente proyecto se supone que:

\begin{enumerate}
    \item El consultorio odontológico \textit{Mundo Dental} cuenta con un registro histórico suficiente de turnos de pacientes, abarcando al menos un año de actividad, y que dichos datos se encuentran disponibles en formato digital legible.
    
    \item Los registros históricos contienen información suficiente y relevante para el análisis, incluyendo datos como fecha del turno, identificación del paciente, y estado de asistencia o ausencia.
    
    \item El personal del consultorio colaborará en la provisión de la información necesaria y en la validación de los resultados obtenidos durante las distintas etapas del proyecto.
    
    \item El desarrollo y procesamiento de los datos se realizarán en un entorno controlado, con los recursos computacionales necesarios para ejecutar los algoritmos de análisis y predicción.
    
    \item Los modelos predictivos a implementar —como \textit{Random Forest}, \textit{Regresión Logística} y \textit{SVM} — serán adecuados para el volumen y tipo de datos disponibles.
    
    \item Las plataformas de mensajería seleccionadas (\textit{WhatsApp} o \textit{Telegram}) disponen de interfaces o \textit{API} accesibles para permitir la integración con el asistente virtual o \textit{chatbot}.
    
    \item El \textit{chatbot} desarrollado será capaz de interactuar con los pacientes de manera autónoma para confirmar o cancelar turnos, y de registrar las respuestas de los usuarios de forma estructurada para su posterior análisis.
    
    \item Se dispone de conectividad estable a internet tanto en el entorno del consultorio como en el servidor donde se alojará la aplicación.
    
    \item El cliente final, representado por {\clientename}, jefa del consultorio, participará activamente en la validación funcional del sistema y en las pruebas de aceptación.
    
    \item Los datos utilizados para el desarrollo y prueba del modelo se mantendrán bajo estricta confidencialidad, respetando las normas de protección de datos personales.
    
    \item El sistema final será implementado en un entorno de \textit{hosting} que soporte la ejecución del modelo predictivo y la operación continua del \textit{chatbot}.
\end{enumerate}


\section{6. Product Backlog}
\label{sec:backlog}

El Product Backlog debe organizarse en cuatro \textbf{\textit{\'{e}picas}} fundamentales del proyecto. Cada \'{e}pica debe contener al menos dos historias de usuario que describan funcionalidades clave.

El Product Backlog debe permitir interpretar cómo será el proyecto y su funcionalidad. Se deben indicar claramente las prioridades entre las historias de usuario y si hay alguna opcional.

Las historias de usuario deben ser breves, claras y medibles, expresando el rol, la necesidad y el propósito de cada funcionalidad. También deben tener una prioridad definida para facilitar la planificación de los sprints.

Cada historia de usuario debe incluir una ponderación en \textit{Story Points}, un número entero que representa el tama\~no relativo de la historia. El criterio para calcular los Story Points debe indicarse explícitamente.

Las historias deben seguir el formato: ``\textit{Como [rol], quiero [tal cosa] para [tal otra cosa]}''.

Las \'{e}picas deben estructurarse de la siguiente forma:

\begin{itemize}
  \item \textbf{\'{E}pica 1}
    \begin{itemize}
      \item HU1
      \item HU2
    \end{itemize}
  \item \textbf{\'{E}pica 2}
    \begin{itemize}
      \item HU3
      \item HU4
    \end{itemize}
  \item \textbf{\'{E}pica 3}
    \begin{itemize}
      \item HU5
      \item HU6
    \end{itemize}
  \item \textbf{\'{E}pica 4}
    \begin{itemize}
      \item HU7
      \item HU8
    \end{itemize}
\end{itemize}

\textbf{Reglas para definir historias de usuario:}
\begin{itemize}
  \item Ser concisas y claras.
  \item Expresarlas en términos cuantificables y medibles.
  \item No dejar margen para interpretaciones ambiguas.
  \item Indicar claramente su prioridad y si son opcionales.
  \item Considerar regulaciones y normas vigentes.
\end{itemize}

\section{7. Criterios de aceptación de historias de usuario}
\label{sec:criteriosAceptacion}

Los criterios de aceptación deben establecerse para cada historia de usuario, asegurando que se cumplan las condiciones necesarias para que la funcionalidad sea validada correctamente.

Cada historia debe tener criterios medibles, específicos y verificables. Deben permitir validar que se cumple con las necesidades del usuario.

Se estructuran de forma análoga a las \'{e}picas del backlog:

\begin{itemize}
  \item \textbf{\'{E}pica 1}
    \begin{itemize}
      \item Criterios de aceptación HU1
      \item Criterios de aceptación HU2
    \end{itemize}
  \item \textbf{\'{E}pica 2}
    \begin{itemize}
      \item Criterios de aceptación HU3
      \item Criterios de aceptación HU4
    \end{itemize}
  \item \textbf{\'{E}pica 3}
    \begin{itemize}
      \item Criterios de aceptación HU5
      \item Criterios de aceptación HU6
    \end{itemize}
  \item \textbf{\'{E}pica 4}
    \begin{itemize}
      \item Criterios de aceptación HU7
      \item Criterios de aceptación HU8
    \end{itemize}
\end{itemize}

\textbf{Reglas para definir criterios de aceptación:}
\begin{itemize}
  \item Medibles y verificables.
  \item Especificar cuándo una historia se considera completada.
  \item Incluir condiciones específicas.
  \item No ambiguos.
  \item Probables de testear funcional o técnicamente.
  \item Mínimo 3 criterios por HU.
\end{itemize}

\section{8. Fases de CRISP-DM}

\begin{enumerate}
  \item \textbf{Comprensión del negocio:} objetivo, valor agregado de IA, métricas de éxito.
  \item \textbf{Comprensión de los datos:} tipo, origen, cantidad, calidad.
  \item \textbf{Preparación de los datos:} características clave, transformaciones necesarias.
  \item \textbf{Modelado:} tipo de problema, algoritmos posibles.
  \item \textbf{Evaluación del modelo:} métricas de rendimiento.
  \item \textbf{Despliegue del modelo (opcional):} tipo de despliegue y herramientas.
\end{enumerate}

\section{9. Desglose del trabajo en tareas}
\label{sec:wbs}

A partir de cada Historia de Usuario (HU) definida en la sección 6, descomponer el trabajo en tareas técnicas concretas, medibles y acotadas en el tiempo.

\begin{itemize}
\item Seleccionar entre 2 y 3 tareas por cada historia de usuario.
\item Cada tarea debe estar claramente formulada, ser técnica, accionable y con una estimación horaria entre 2 y 8 horas.
\item Evitar tareas genéricas (como ''desarrollar funcionalidad´´) o demasiado amplias.
\item Si una tarea supera las 8 horas, debe dividirse en subtareas.
\item Indicar la prioridad relativa de cada tarea (Alta, Media o Baja).
\end{itemize}

\begin{table}[htpb]
\centering
\begin{tabularx}{\linewidth}{@{}|X|X|c|c|@{}}
\hline
\rowcolor[HTML]{C0C0C0}
Historia de usuario & Tarea técnica & Estimación & Prioridad \\ \hline
HU1 & Tarea 1 HU1 & 6 h & Alta \\ \hline
HU1 & Tarea 2 HU1 & 8 h & Alta \\ \hline
HU2 & Tarea 1 HU2 & 5 h & Media \\ \hline
HU2 & Tarea 2 HU2 & 6 h & Alta \\ \hline
... & ... & ... & ... \\ \hline
\end{tabularx}
\end{table}

\textbf{Criterios para estimar tiempos:}
\begin{itemize}
  \item Considerar la complejidad técnica, el nivel de incertidumbre y la experiencia previa.
  \item Evitar subestimar el esfuerzo: estimar el tiempo realista que llevaría implementar, testear y documentar cada tarea.
  \item Basar la estimación en la experiencia propia o en referencias de tareas similares.
\end{itemize}

\textbf{Sobre la prioridad:}
\begin{itemize}
  \item Asignar una prioridad relativa (Alta, Media o Baja) según la relevancia funcional de la tarea y su impacto en los entregables.
  \item Priorizar tareas que estén vinculadas a criterios de aceptación de las HU o que sean necesarias para desbloquear otras.
  \item Incluir tareas opcionales solo si están bien justificadas.
\end{itemize}

\textbf{Recomendaciones generales:}
\begin{itemize}
  \item -Enfocarse en tareas que surgen directamente de las HU planteadas.
  \item No es necesario cubrir las 600 horas del proyecto en esta sección: el foco está en el desglose de funcionalidades clave.
  \item Este trabajo será la base para organizar algunos de los sprints y elaborar el cronograma del proyecto, por lo que debe ser claro y realista.
\end{itemize}

\section{10. Planificación de Sprints}

Organizar las tareas técnicas del proyecto en sprints de trabajo que permitan distribuir de forma equilibrada la carga horaria total, estimada en 600 horas.

\textbf{Consigna:}
\begin{itemize}
  \item Completar una tabla que relacione sprints con HU y tareas técnicas correspondientes.
  \item Incluir estimación en horas para cada tarea.
  \item Indicar responsable y porcentaje de avance estimado o completado.
  \item Contemplar también tareas de planificación, documentación, redacción de memoria y preparación de defensa.
\end{itemize}

\textbf{Conceptos clave:}
\begin{itemize}
  \item Una \'{e}pica es una unidad funcional amplia; una historia de usuario es una funcionalidad concreta; un sprint es una unidad de tiempo donde se ejecutan tareas.
  \item Las tareas son el nivel más desagregado: permiten estimar tiempos, asignar responsables y monitorear progreso.
\end{itemize}

\textbf{Duración sugerida:}
\begin{itemize}
  \item Para un proyecto de 600 h, se recomienda planificar entre 10 y 12 sprints de aproximadamente 2 semanas cada uno.
  \item Asignar entre 45 y 50 horas efectivas por sprint a tareas técnicas.
  \item Reservar 100 a 120 h para actividades no técnicas (planificación, escritura, reuniones, defensa).
\end{itemize}

\textbf{Importante:}
\begin{itemize}
  \item En proyectos individuales, el responsable suele ser el propio autor.
  \item Aun así, desagregar tareas facilita el seguimiento y mejora continua.
\end{itemize}

\textbf{Conversión opcional de Story Points a horas:}
\begin{itemize}
  \item 1 SP \(\approx\) 2 h como referencia flexible.
  \item Tener en cuenta aproximaciones tipo Fibonacci.
\end{itemize}

\begin{table}[htpb]
\centering
\caption{Formato sugerido}
\begin{tabularx}{\linewidth}{@{}|l|l|X|c|l|c|@{}}
\hline
\rowcolor[HTML]{C0C0C0}
Sprint & HU o fase & Tarea & Horas / SP & Responsable & \% Completado \\ \hline
Sprint 0 & Planificación & Definir alcance y cronograma & 10 h & Alumno & 100\% \\ \hline
Sprint 0 & Planificación & Reunión con el tutor/cliente & 5 h & Alumno & 50\% \\ \hline
Sprint 0 & Planificación & Ajuste de los entregables & 6 h & Alumno & 25\% \\ \hline
Sprint 1 & HU1 & Tarea 1 HU1 & 6 h / 3 SP & Alumno & 0\% \\ \hline
Sprint 1 & HU1 & Tarea 2 HU1 & 10 h / 5 SP & Alumno & 0\% \\ \hline
Sprint 2 & HU2 & Tarea 1 HU2 & 7 h / 5 SP & Alumno & 0\% \\ \hline
... & ... & ... & ... & ... & ... \\ \hline
Sprint 5 & Escritura & Redacción memoria & 50 h / 34 SP & Alumno & 0\% \\ \hline
Sprint 6 & Defensa & Preparación de la exposición & 20 h / 13 SP & Alumno & 0\% \\ \hline
\end{tabularx}
\end{table}

\textbf{Recomendaciones:}
\begin{itemize}
  \item Verificar que la carga horaria por sprint sea equilibrada.
  \item Usar sprints de 1 a 3 semanas, acordes al cronograma general.
  \item Actualizar el \% completado durante el seguimiento del proyecto.
  \item Considerar un sprint final exclusivo para pruebas, revisión y ajustes antes de la defensa.
\end{itemize}


\section{11. Diagrama de Gantt (sprints)}
\label{sec:gantt}

Visualizar en un diagrama de Gantt la planificación temporal del proyecto, tomando como base los sprints definidos en la sección anterior. Debe contemplar todas las horas del proyecto.

Consigna:

\begin{itemize}

\item Elaborar un diagrama de Gantt que muestre la secuencia temporal de los sprints.

\item Cada fila debe representar un sprint (con su número o nombre), y el eje horizontal debe indicar el tiempo (en semanas o fechas concretas).

\item Las tareas técnicas derivadas de HU deben diferenciarse visualmente (por ejemplo, con un color distinto) de las tareas no técnicas (planificación, redacción, defensa).

\item Incluir todas las tareas estimadas en cada sprint.
\end{itemize}

Recomendaciones para el Gantt:

\begin{itemize}

	\item Podés usar herramientas gratuitas como TeamGantt, ClickUp, GanttProject, [Google Sheets], [Trello + Planyway], entre otras.
	\item Ordená los sprints de forma cronológica, comenzando con Sprint 0 (planificación) y finalizando con el sprint de defensa.
	\item Asegurate de reflejar la duración realista de cada sprint según tu disponibilidad y el cronograma general del posgrado.
	\item Incluí hitos importantes: reuniones, entregas parciales, defensa.
\end{itemize}


Incluir una imagen legible del diagrama de Gantt. Si es muy ancho, presentar primero la tabla y luego el gráfico de barras.



\section{12. Gobernanza de datos}
\label{sec:gobernanza}

En esta sección se debe analizar de manera integral el marco normativo y ético asociado al uso de los datos y al desarrollo de soluciones basadas en inteligencia artificial.

El análisis se divide en dos partes:

\begin{itemize}
  \item Cumplimiento normativo.
  \item Ética en el uso de inteligencia artificial.
\end{itemize}

\subsection{Cumplimiento normativo}
Este análisis es clave para garantizar el cumplimiento normativo y evitar conflictos legales durante el desarrollo y publicación del proyecto.

En esta subsección se debe identificar si los datos utilizados en el proyecto están sujetos a normativas de protección de datos y privacidad, y en qué condiciones pueden emplearse.

Aspectos a considerar:

\begin{itemize}
  \item Determinar si los datos están regulados por normativas como el GDPR, la Ley 25.326 de Protección de Datos Personales (Argentina), la HIPAA u otras, según la jurisdicción o la temática del proyecto.
  \item Aclarar si las fuentes de datos son propias, de acceso público o licenciadas. En este último caso, explicitar la licencia correspondiente y sus condiciones de uso. Realizar este mismo análisis para las herramientas y, en caso de corresponder, para los modelos preentrenados a emplear.
  \item Analizar la viabilidad legal del proyecto, considerando los mecanismos necesarios para garantizar el cumplimiento normativo y la trazabilidad de los datos.
\end{itemize}

\subsection{Ética en el uso de inteligencia artificial}

En esta subsección se debe reflexionar sobre los aspectos éticos vinculados al uso de datos y algoritmos de inteligencia artificial. Se espera una evaluación de la integridad del sistema y su impacto social.

Aspectos a considerar:

\begin{itemize}
	\item Identificar posibles sesgos en los datos o modelos (ideológicos, culturales, de género, geográficos, etc.) y analizar sus consecuencias (por ejemplo: discriminación, manipulación, pérdida de privacidad o desinformación).
	\item Analizar los posibles riesgos en términos de impacto social del uso indebido de la solución a desarrollar.
	\item En caso de corresponder, proponer medidas de mitigación y mecanismos de confianza (auditorías, documentación, trazabilidad, revisión humana, etc.).
	\item Finalmente, elaborar una reflexión general sobre la ética del proyecto, considerando tanto la equidad y transparencia del sistema como su impacto potencial en la sociedad.

\end{itemize}


\section{13. Gestión de riesgos}
\label{sec:riesgos}

\begin{consigna}{red}
a) Identificación de los riesgos (al menos cinco) y estimación de sus consecuencias:
 
Riesgo 1: detallar el riesgo (riesgo es algo que si ocurre altera los planes previstos de forma negativa)
\begin{itemize}
	\item Severidad (S): mientras más severo, más alto es el número (usar números del 1 al 10).\\
	Justificar el motivo por el cual se asigna determinado número de severidad (S).
	\item Probabilidad de ocurrencia (O): mientras más probable, más alto es el número (usar del 1 al 10).\\
	Justificar el motivo por el cual se asigna determinado número de (O). 
\end{itemize}   

Riesgo 2:
\begin{itemize}
	\item Severidad (S): X.\\
	Justificación...
	\item Ocurrencia (O): Y.\\
	Justificación...
\end{itemize}

Riesgo 3:
\begin{itemize}
	\item Severidad (S):  X.\\
	Justificación...
	\item Ocurrencia (O): Y.\\
	Justificación...
\end{itemize}


b) Tabla de gestión de riesgos:      (El RPN se calcula como RPN=SxO)

\begin{table}[htpb]
\centering
\begin{tabularx}{\linewidth}{@{}|X|c|c|c|c|c|c|@{}}
\hline
\rowcolor[HTML]{C0C0C0} 
Riesgo & S & O & RPN & S* & O* & RPN* \\ \hline
       &   &   &     &    &    &      \\ \hline
       &   &   &     &    &    &      \\ \hline
       &   &   &     &    &    &      \\ \hline
       &   &   &     &    &    &      \\ \hline
       &   &   &     &    &    &      \\ \hline
\end{tabularx}%
\end{table}

Criterio adoptado: 

Se tomarán medidas de mitigación en los riesgos cuyos números de RPN sean mayores a...

Nota: los valores marcados con (*) en la tabla corresponden luego de haber aplicado la mitigación.

c) Plan de mitigación de los riesgos que originalmente excedían el RPN máximo establecido:
 
Riesgo 1: plan de mitigación (si por el RPN fuera necesario elaborar un plan de mitigación).
  Nueva asignación de S y O, con su respectiva justificación:
  \begin{itemize}
	\item Severidad (S*): mientras más severo, más alto es el número (usar números del 1 al 10).
          Justificar el motivo por el cual se asigna determinado número de severidad (S).
	\item Probabilidad de ocurrencia (O*): mientras más probable, más alto es el número (usar del 1 al 10).
          Justificar el motivo por el cual se asigna determinado número de (O).
	\end{itemize}

Riesgo 2: plan de mitigación (si por el RPN fuera necesario elaborar un plan de mitigación).
 
Riesgo 3: plan de mitigación (si por el RPN fuera necesario elaborar un plan de mitigación).

\end{consigna}

\section{14. Sprint Review}
\label{sec:sprint_review}

La revisión de sprint (\emph{Sprint Review}) es una práctica fundamental en metodologías ágiles. Consiste en revisar y evaluar lo que se ha completado al finalizar un sprint. En esta instancia, se presentan los avances y se verifica si las funcionalidades cumplen con los criterios de aceptación establecidos. También se identifican entregables parciales y se consideran ajustes si es necesario.

Aunque el proyecto aún se encuentre en etapa de planificación, esta sección permite proyectar cómo se evaluarán las funcionalidades más importantes del backlog. Esta mirada anticipada favorece la planificación enfocada en valor y permite reflexionar sobre posibles obstáculos.

\textbf{Objetivo:} anticipar cómo se evaluará el avance del proyecto a medida que se desarrollen las funcionalidades, utilizando como base al menos cuatro historias de usuario del \emph{Product Backlog}.


Seleccionar al menos 4 HU del Product Backlog. Para cada una, completar la siguiente tabla de revisión proyectada:

\textbf{Formato sugerido:}
\begin{table}[htpb]
\renewcommand{\arraystretch}{1.5}
\begin{tabular}{|>{\raggedright\arraybackslash}m{2.4cm}|
                >{\raggedright\arraybackslash}m{2.3cm}|
                >{\raggedright\arraybackslash}m{3cm}|
                >{\raggedright\arraybackslash}m{3cm}|
                >{\raggedright\arraybackslash}m{3cm}|}
\hline
\rowcolor[HTML]{CCCCCC}
\textbf{HU seleccionada} & \textbf{Tareas asociadas} & \textbf{Entregable esperado} & \textbf{¿Cómo sabrás que está cumplida?} & \textbf{Observaciones o riesgos} \\
\hline
                         & Tarea 1 &                             &                                           &                                     \\ \cline{2-2}
\multirow{-2}{=}{HU1}    & Tarea 2 & \multirow{-2}{=}{Módulo funcional} & \multirow{-2}{=}{Cumple criterios de aceptación definidos} & \multirow{-2}{=}{Falta validar con el tutor} \\
\hline
                         & Tarea 1 &                             &                                           &                                     \\ \cline{2-2}
\multirow{-2}{=}{HU3}    & Tarea 2 & \multirow{-2}{=}{Reporte generado} & \multirow{-2}{=}{Exportación disponible y clara} & \multirow{-2}{=}{Requiere datos reales} \\
\hline
                         & Tarea 1 &                             &                                           &                                     \\ \cline{2-2}
\multirow{-2}{=}{HU5}    & Tarea 2 & \multirow{-2}{=}{Panel de gestión} & \multirow{-2}{=}{Roles diferenciados operativos} & \multirow{-2}{=}{Riesgo en integración} \\
\hline
                         & Tarea 1 &                             &                                           &                                     \\ \cline{2-2}
\multirow{-2}{=}{HU7}    & Tarea 2 & \multirow{-2}{=}{Informe trimestral} & \multirow{-2}{=}{PDF con gráficos y evolución} & \multirow{-2}{=}{Puede faltar tiempo para ajustes} \\
\hline
\end{tabular}
\end{table}

\section{15. Sprint Retrospective}    
\label{sec:sprint_retro}

La retrospectiva de sprint es una práctica orientada a la mejora continua. Al finalizar un sprint, el equipo (o el alumno, si trabaja de forma individual) reflexiona sobre lo que funcionó bien, lo que puede mejorarse y qué acciones concretas pueden implementarse para trabajar mejor en el futuro.

Durante la cursada se propuso el uso de la \textbf{Estrella de la Retrospectiva}, que organiza la reflexión en torno a cinco ejes:

\begin{itemize}
\item  ¿Qué hacer más?
\item  ¿Qué hacer menos?
\item  ¿Qué mantener?
\item  ¿Qué empezar a hacer?
\item  ¿Qué dejar de hacer?
\end{itemize}

Aun en una etapa temprana, esta herramienta permite que el alumno planifique su forma de trabajar, identifique anticipadamente posibles dificultades y diseñe estrategias de organización personal.

\textbf{Objetivo:} reflexionar sobre las condiciones iniciales del proyecto, identificando fortalezas, posibles dificultades y estrategias de mejora, incluso antes del inicio del desarrollo.


Completar la siguiente tabla tomando como referencia los cinco ejes de la Estrella de la Retrospectiva (\emph{Starfish} o estrella de mar). Esta instancia te ayudará a definir buenas prácticas desde el inicio y prepararte para enfrentar el trabajo de forma organizada y flexible. Se deberá completar la tabla al menos para 3 sprints técnicos y 1 no técnico.

\textbf{Formato sugerido:}

\begin{table}[htpb]
\renewcommand{\arraystretch}{1.4}
\begin{tabular}{|>{\raggedright\arraybackslash}p{1.8cm}|
                >{\raggedright\arraybackslash}p{2.3cm}|
                >{\raggedright\arraybackslash}p{2.3cm}|
                >{\raggedright\arraybackslash}p{2.3cm}|
                >{\raggedright\arraybackslash}p{2.3cm}|
                >{\raggedright\arraybackslash}p{2.3cm}|}
\hline
\rowcolor[HTML]{CCCCCC} 
\textbf{Sprint tipo y N°} & \textbf{¿Qué hacer más?} & \textbf{¿Qué hacer menos?} & \textbf{¿Qué mantener?} & \textbf{¿Qué empezar a hacer?} & \textbf{¿Qué dejar de hacer?} \\
\hline
Sprint técnico - 1 & Validaciones continuas con el alumno & Cambios sin versión registrada & Pruebas con datos simulados & Documentar cambios propuestos & Ajustes sin análisis de impacto \\
\hline
Sprint técnico - 2 & Verificar configuraciones en múltiples escenarios & Modificar parámetros sin guardar historial & Perfiles reutilizables & Usar logs para configuración & Repetir pruebas manuales innecesarias \\
\hline
Sprint técnico - 8 & Comparar correlaciones con casos previos & Cambiar parámetros sin justificar & Revisión cruzada de métricas & Anotar configuraciones usadas & Trabajar sin respaldo de datos \\
\hline
Sprint no técnico - 12 (por ej.: ``Defensa'') & Ensayos orales con feedback & Cambiar contenidos en la memoria & Material visual claro & Dividir la presentación por bloques & Agregar gráficos difíciles de explicar \\
\hline
\end{tabular}
\end{table}




\end{document}